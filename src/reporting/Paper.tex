% Options for packages loaded elsewhere
\PassOptionsToPackage{unicode}{hyperref}
\PassOptionsToPackage{hyphens}{url}
%
\documentclass[
]{article}
\usepackage{amsmath,amssymb}
\usepackage{iftex}
\ifPDFTeX
  \usepackage[T1]{fontenc}
  \usepackage[utf8]{inputenc}
  \usepackage{textcomp} % provide euro and other symbols
\else % if luatex or xetex
  \usepackage{unicode-math} % this also loads fontspec
  \defaultfontfeatures{Scale=MatchLowercase}
  \defaultfontfeatures[\rmfamily]{Ligatures=TeX,Scale=1}
\fi
\usepackage{lmodern}
\ifPDFTeX\else
  % xetex/luatex font selection
\fi
% Use upquote if available, for straight quotes in verbatim environments
\IfFileExists{upquote.sty}{\usepackage{upquote}}{}
\IfFileExists{microtype.sty}{% use microtype if available
  \usepackage[]{microtype}
  \UseMicrotypeSet[protrusion]{basicmath} % disable protrusion for tt fonts
}{}
\makeatletter
\@ifundefined{KOMAClassName}{% if non-KOMA class
  \IfFileExists{parskip.sty}{%
    \usepackage{parskip}
  }{% else
    \setlength{\parindent}{0pt}
    \setlength{\parskip}{6pt plus 2pt minus 1pt}}
}{% if KOMA class
  \KOMAoptions{parskip=half}}
\makeatother
\usepackage{xcolor}
\usepackage[margin=1in]{geometry}
\usepackage{longtable,booktabs,array}
\usepackage{calc} % for calculating minipage widths
% Correct order of tables after \paragraph or \subparagraph
\usepackage{etoolbox}
\makeatletter
\patchcmd\longtable{\par}{\if@noskipsec\mbox{}\fi\par}{}{}
\makeatother
% Allow footnotes in longtable head/foot
\IfFileExists{footnotehyper.sty}{\usepackage{footnotehyper}}{\usepackage{footnote}}
\makesavenoteenv{longtable}
\usepackage{graphicx}
\makeatletter
\newsavebox\pandoc@box
\newcommand*\pandocbounded[1]{% scales image to fit in text height/width
  \sbox\pandoc@box{#1}%
  \Gscale@div\@tempa{\textheight}{\dimexpr\ht\pandoc@box+\dp\pandoc@box\relax}%
  \Gscale@div\@tempb{\linewidth}{\wd\pandoc@box}%
  \ifdim\@tempb\p@<\@tempa\p@\let\@tempa\@tempb\fi% select the smaller of both
  \ifdim\@tempa\p@<\p@\scalebox{\@tempa}{\usebox\pandoc@box}%
  \else\usebox{\pandoc@box}%
  \fi%
}
% Set default figure placement to htbp
\def\fps@figure{htbp}
\makeatother
\setlength{\emergencystretch}{3em} % prevent overfull lines
\providecommand{\tightlist}{%
  \setlength{\itemsep}{0pt}\setlength{\parskip}{0pt}}
\setcounter{secnumdepth}{-\maxdimen} % remove section numbering
\usepackage{bookmark}
\IfFileExists{xurl.sty}{\usepackage{xurl}}{} % add URL line breaks if available
\urlstyle{same}
\hypersetup{
  pdftitle={IMDb Runtime \& Ratings Analysis (2011--2020)},
  pdfauthor={Group 7},
  hidelinks,
  pdfcreator={LaTeX via pandoc}}

\title{IMDb Runtime \& Ratings Analysis (2011--2020)}
\author{Group 7}
\date{2025-10-09}

\begin{document}
\maketitle

{
\setcounter{tocdepth}{2}
\tableofcontents
}
\subsection{Motivation}\label{motivation}

In today's world, movie ratings play an integral part when viewers
decide which movie to watch. Among the many factors that influence the
rating of a movie, runtime stands out as a relevant one when it comes to
audience evaluation. Previously, a study by Choudhary et al.~(2024)
found runtime to be a statistically significant variable, influencing
the rating of a movie across different genres, though the magnitude of
this effect varied between genres. This raises the question about
exploring not only the magnitude, but also the direction of the runtime
effect. On the one hand, longer movies allow for more complex
storytelling and character development, but on the other hand,
decreasing attention spans (Hayes, 2024) among people may actually drive
longer movie ratings down.

Previous research suggests a significant relationship between movie
genres and their ratings, however often note that isolation of the genre
variable is not clearly attainable (Choudhary et al., 2024). While there
is no clear consensus in the literature on which genres consistently
receive the highest ratings, Matthews (2021) argues that audience
expectations are more structured for mainstream genres such as drama,
action, and comedy than for niche categories. In these popular genres,
ratings often reflect not only film quality but also how well a movie
aligns with established genre conventions. Building on this, the present
research focuses specifically on adventure, action, and comedy, as these
genres have consistently generated the highest box office revenues in
North America over the past decades (Statista, 2025). Adventure and
action dominate due to their large-scale productions, international
appeal, and substantial commercial success, while comedy plays a
complementary role by reflecting cultural preferences and offering
insights into audience diversity. Together, these genres allow for a
balanced analysis that captures both the financial strength of
blockbuster categories and the cultural relevance of humor-driven films.

In addition to examining the relationship between runtime, genre, and
IMDb ratings, it is also important to define a clear time interval for
the research. The global Netflix subscriber data provides a useful
benchmark: in early 2013, Netflix had around 30 million paid
subscribers, but by 2015 that number had more than doubled to over 70
million (Netflix, 2025). This rapid acceleration reflects the point at
which streaming shifted from an emerging model to a mainstream mode of
media consumption, making 2015 a suitable cutoff year. Streaming
services mark a natural cutoff point because they changed how runtimes
are perceived: before streaming, films were optimized for theatrical
showings and ticket sales, while streaming enabled more flexibility in
length. As audiences gained on-demand access, tolerance for both shorter
and longer runtimes shifted, making streaming adoption a key turning
point for analyzing runtime effects on IMDb ratings. To ensure balanced
group sizes around our cutoff, we divided the data into two equal-length
periods: 2011--2015 and 2016--2020. This allows us to compare
runtime--rating relationships across time while keeping the number of
films per period reasonably similar. Accordingly, the research will
compare films released between 2011 and 2015 with those released between
2016 and 2020.

Thus, the research question for this project is defined as ``How does
runtime influence IMDb ratings, and how is this relationship moderated
by genre (Adventure, Action and Comedy) when comparing films released
between 2011 and 2015 to those released between 2016 and 2020?'' This
research addresses a gap in the currently existing literature of factors
that influence audience reception of movies such as IMDb ratings by
investigating how the release year (between 2011 and 2015 versus between
2016 and 2020) and genre (Adventure, Action and Comedy) have a
moderating effect on the relationship between runtime and IMDb ratings
of movies. Previous research looked at the individual effects of
runtime, release year and genre (Horror, Comedy and Action) on IMDb
ratings while this research also looks at the relative effects of these
variables and how they interact with each other.

Further, this research is relevant to different marketing stakeholders
in the movie industry such as marketing managers of movie studios,
streaming platforms and cinemas. By providing valuable insights on the
runtime preferences of audiences these can be used for example by movie
studios to create movies across different genres with an optimal runtime
and by streaming platforms and cinemas to find the optimal marketing
strategy to movies of different runtimes and genres.

\subsection{Data}\label{data}

\begin{itemize}
\tightlist
\item
  We are using two files from the IMDB database:

  \begin{enumerate}
  \def\labelenumi{\arabic{enumi})}
  \tightlist
  \item
    title.basics.tsv.gz (contains unique identifier of the title, the
    type/format of the title, original title, the release year of a
    title, TV Series end year, primary runtime of the title, genres)
  \item
    title.ratings.tsv.gz (contains unique identifier of the title,
    averageRating and number of votes the title has received)
  \end{enumerate}
\item
  The final dataset includes 20144 observations. A variable description
  / operationalisation table is below.
\end{itemize}

\subsection{Variable Description and
Operationalisation}\label{variable-description-and-operationalisation}

\begin{longtable}[]{@{}
  >{\raggedright\arraybackslash}p{(\linewidth - 6\tabcolsep) * \real{0.1232}}
  >{\raggedright\arraybackslash}p{(\linewidth - 6\tabcolsep) * \real{0.1377}}
  >{\raggedright\arraybackslash}p{(\linewidth - 6\tabcolsep) * \real{0.1522}}
  >{\raggedright\arraybackslash}p{(\linewidth - 6\tabcolsep) * \real{0.5870}}@{}}
\toprule\noalign{}
\begin{minipage}[b]{\linewidth}\raggedright
Variable
\end{minipage} & \begin{minipage}[b]{\linewidth}\raggedright
Type
\end{minipage} & \begin{minipage}[b]{\linewidth}\raggedright
Source
\end{minipage} & \begin{minipage}[b]{\linewidth}\raggedright
Operationalisation (How it is measured/defined)
\end{minipage} \\
\midrule\noalign{}
\endhead
\bottomrule\noalign{}
\endlastfoot
\texttt{Rating} & Dependent & \texttt{title.ratings} &
\texttt{averageRating} (on a scale of 1-10, continuous) \\
\texttt{Runtime10} & Independent & \texttt{title.basics} &
(\texttt{runtimeMinutes} - mean(\texttt{runtimeMinutes}))/10 \\
\texttt{Genre} & Moderator & \texttt{title.basics} & Comedy (reference),
Adventure (dummy), Action (dummy) \\
\texttt{YearGroup} & Moderator & \texttt{title.basics} & 2011-2015
(reference), 2016-2020 (dummy) \\
\texttt{numVotes} & Control & \texttt{title.rating} & Total IMDb
votes \\
\texttt{logVotes} & Control & Derived & log10(\texttt{numVotes}).
Interpreted as +1 = 10x more votes \\
\texttt{Intercept} & Derived & Model & Comedy movie, 2011-2015, mean
\texttt{runtimeMinutes}, mean \texttt{logVotes} \\
\end{longtable}

To ensure data quality and meaningful analysis, we applied the following
filters to the dataset:

\begin{enumerate}
\def\labelenumi{\arabic{enumi})}
\tightlist
\item
  Runtime filter: We excluded movies with a runtime below 30 minutes.
\end{enumerate}

Rationale: Very short films (e.g., shorts, experimental pieces) are
structurally different from feature-length movies, and including them
would bias our analysis of how runtime affects ratings.

\begin{enumerate}
\def\labelenumi{\arabic{enumi})}
\setcounter{enumi}{1}
\tightlist
\item
  Vote count filter: We excluded movies with fewer than 50 votes.
\end{enumerate}

Rationale: IMDb ratings for movies with very few votes are often
unstable and unreliable. Setting a threshold of 50 votes ensures that
our dataset contains movies with sufficient audience engagement to
provide a more representative measure of audience opinion.

\subsubsection{Handling Missing Values
(Runtime)}\label{handling-missing-values-runtime}

During the data preparation phase, we observed that some movies in the
IMDb dataset had missing values for runtimeMinutes. Further exploration
revealed that these missing values were not randomly distributed: they
tended to occur more frequently in certain genres and time periods.
Because runtime is our main independent variable, we could not simply
drop these movies, as that might bias our analysis.

To address this, we applied a median imputation strategy grouped by
genre and release period. Specifically: Movies were grouped by genre
(Comedy, Action, Adventure) and release year group (2011--2015
vs.~2016--2020). Within each group, missing runtimes were replaced with
the median runtime of that group. (Median was chosen instead of the mean
because it is more robust to outliers.) This approach ensures that our
dataset remains complete without artificially inflating runtimes or
discarding a large number of observations.

\subsection{Method}\label{method}

For this research we will perform a multiple linear regression with
interaction terms to find ou t whether the runtime of a movie
(continuous) influences its IMDb rating (continuous) and whether a
film's genre (Comedy, Adventure or Action) (categorical) and the release
period (2011-2015 vs.~2016-2020) (categorical) influence this
relationship. The runtime is the independent variable, the IMDb rating
is the dependent variable and the two moderators are Genre (Comedy
vs.~Adventure vs.~Action) and Release Period (2011-2015 vs.~2016-2020).
Further, we will include the number of IMDb votes (expressed as a
log-scaled variable) as a control variable since ratings based on more
votes are usually more stable and reliable (Xie \& Lui, 2013). We have
chosen for a multiple linear regression with interaction terms as this
is the most suitable way to combine these variable types, a continuous
independent and dependent variable and two categorical moderators, into
one model. This will lead to the following model: Rating = X₀ +
β₁·Runtime10 + β₂·Adventure + β₃·Action + β₄·Yeargroup2016--2020 +
β₅·(Runtime10 × Adventure) + β₆·(Runtime10 × Action) + β₇·(Runtime10 ×
Yeargroup2016--2020) + β₈·log₁₀(Votes) + ϵ

where:\\
- β₁ = How the effect of +10 minutes runtime on IMDb ratings changes for
Comedy movies released in 2011--2015\\
- β₂, β₃ = How the effect of +10 minutes runtime on IMDb ratings changes
for Adventure and Action movies compared to Comedy movies\\
- β₄ = How the IMDb ratings change between movies released in 2016--2020
compared to movies released in 2011--2015\\
- β₅, β₆ = How the effect of +10 minutes runtime on IMDb ratings changes
for Adventure and Action movies compared to Comedy movies\\
- β₇ = How the effect of +10 minutes runtime on IMDb ratings changes for
movies released in 2016--2020 compared to 2011--2015\\
- β₈ = How the IMDb ratings change between movies when the number of
votes increases by a factor of 10

We began with a descriptive analysis to explore the data. This included
examining trends over time, looking at the distributions of ratings,
votes (on a log scale), and runtimes, and summarizing results by both
year and genre (focusing on Action, Comedy, and Adventure). We conducted
the analysis in two steps:

\begin{itemize}
\item
  \textbf{Descriptive analysis:}\\
  We first explored the data by examining trends over time, the
  distributions of IMDb ratings, votes (on a log scale), and runtimes,
  and summary statistics by both year and genre (Action, Comedy, and
  Adventure).
\item
  \textbf{Regression analysis:}\\
  We then estimated an ordinary least squares (OLS) regression with
  interaction terms to test whether a film's runtime influences its IMDb
  rating, and whether this relationship differs by genre and release
  period.
\end{itemize}

To control for rating stability, we included the number of IMDb votes
(log₁₀-transformed) as a covariate. Robust HC3 standard errors were used
to account for heteroskedasticity, and we checked diagnostics including
heteroskedasticity tests, residual normality, and variance inflation
factors (VIF).

\subsection{Descriptive Results}\label{descriptive-results}

\begin{longtable}[]{@{}lrrrrr@{}}
\caption{Summary by Genre}\tabularnewline
\toprule\noalign{}
genres & n\_films & mean\_rating & sd\_rating & mean\_votes &
mean\_runtime \\
\midrule\noalign{}
\endfirsthead
\toprule\noalign{}
genres & n\_films & mean\_rating & sd\_rating & mean\_votes &
mean\_runtime \\
\midrule\noalign{}
\endhead
\bottomrule\noalign{}
\endlastfoot
Action & 6128 & 5.45 & 1.45 & 24532.06 & 105.92 \\
Adventure & 3480 & 5.68 & 1.43 & 32668.91 & 96.61 \\
Comedy & 14597 & 5.61 & 1.25 & 7798.29 & 99.22 \\
\end{longtable}

\begin{longtable}[]{@{}rrrrr@{}}
\caption{Summary by Year}\tabularnewline
\toprule\noalign{}
startYear & n\_films & mean\_rating & mean\_votes & mean\_runtime \\
\midrule\noalign{}
\endfirsthead
\toprule\noalign{}
startYear & n\_films & mean\_rating & mean\_votes & mean\_runtime \\
\midrule\noalign{}
\endhead
\bottomrule\noalign{}
\endlastfoot
2011 & 1716 & 5.61 & 17710.15 & 99.33 \\
2012 & 1854 & 5.65 & 15826.17 & 99.58 \\
2013 & 1995 & 5.65 & 15101.23 & 100.10 \\
2014 & 2135 & 5.63 & 15745.76 & 100.54 \\
2015 & 2124 & 5.61 & 11960.07 & 100.94 \\
2016 & 2302 & 5.58 & 11425.89 & 100.68 \\
2017 & 2288 & 5.62 & 10760.73 & 101.70 \\
2018 & 2336 & 5.56 & 10126.44 & 101.51 \\
2019 & 2300 & 5.57 & 9580.40 & 101.14 \\
2020 & 1639 & 5.45 & 6668.85 & 97.17 \\
\end{longtable}

\subsubsection{Descriptive figures}\label{descriptive-figures}

\paragraph{Film output over time}\label{film-output-over-time}

The number of films released annually increased steadily from 2011 until
around 2018, reaching a peak of over 2,000 titles per year. After 2018,
we observe a decline in output, with a particularly sharp drop in 2020.
This is likely due to the disruptions in global film production during
the COVID-19 pandemic.

\paragraph{Runtime patterns}\label{runtime-patterns}

Since runtime is our main explanatory variable, we first explore its
distribution across time and genres. This helps establish whether there
is enough variation in runtimes to meaningfully relate them to ratings.

\subparagraph{Average runtime per year}\label{average-runtime-per-year}

The average runtime of films remained remarkably consistent across the
2011--2020 period, fluctuating only within a narrow range of about
97--102 minutes. This stability suggests that runtime norms were fairly
entrenched across the decade.

\subparagraph{Distribution of runtimes}\label{distribution-of-runtimes}

Most films fell into a range of 80--120 minutes, with a clear peak
around 95--105 minutes. Only a small share of films were much shorter or
longer, indicating that runtimes cluster strongly around conventional
feature-length standards.

\subparagraph{Runtime distribution by
genre}\label{runtime-distribution-by-genre}

When broken down by genre, Action films showed the highest median
runtimes and the widest spread, reflecting the blockbuster tendency
toward longer movies. Adventure films were slightly shorter but still
above 95 minutes on average, while Comedies were noticeably shorter and
less variable, clustering tightly around the 90--100 minute range.

\paragraph{IMDb ratings (main outcome)}\label{imdb-ratings-main-outcome}

Ratings are the key dependent variable of the research. By describing
their distribution across years and genres, we can see whether patterns
emerge that runtime might help explain.

\subparagraph{Average rating per year}\label{average-rating-per-year}

Mean IMDb ratings across all films were stable at around 5.6 for most of
the decade. However, after 2017 a modest decline is visible, suggesting
either changes in the kinds of films being produced or shifts in
audience evaluation.

\subparagraph{Distribution of IMDb
ratings}\label{distribution-of-imdb-ratings}

The histogram shows that most films score between 5 and 6, with
relatively few extreme ratings. Very high ratings (above 8) are rare,
which reflects IMDb's tendency toward a compressed rating scale centered
in the mid-range.

\subparagraph{Average rating by genre}\label{average-rating-by-genre}

Adventure films achieved the highest average ratings (≈5.7), slightly
above Comedy and Action. Comedies and Action movies clustered more
closely together (≈5.5 to 5.6), indicating that modest genre differences
in audience evaluations exist.

\subparagraph{Heatmap: average IMDb rating per Year x
Genre}\label{heatmap-average-imdb-rating-per-year-x-genre}

Across time, all three genres showed stable ratings without major
trends. Adventure films consistently outperformed the others, while
Action films slightly underperformed, but the differences were small. No
single year shows any significant spikes or drops for any genre.

\paragraph{Votes (control variable)}\label{votes-control-variable}

Finally, we consider votes, which serve as a control variable in the
regression analysis. Audience size matters, because ratings based on
larger numbers of votes are seen as more stable and reliable.

\subparagraph{Distribution of votes (log
scale)}\label{distribution-of-votes-log-scale}

The vote distribution is highly right-skewed: the majority of films
received only a few hundred or thousand votes, while a small number
attracted hundreds of thousands or even millions. This reflects the
blockbuster vs.~niche divide in audience reach.

\subparagraph{Average number of votes by
genre}\label{average-number-of-votes-by-genre}

Adventure films were the most popular in terms of audience engagement,
receiving the highest average vote counts. Action films also attracted
more attention and votes, while Comedies had fewer votes on average,
which might suggest smaller or more fragmented audiences.

\subsection{Regression Results}\label{regression-results}

\subsubsection{Model assumptions}\label{model-assumptions}

\begin{itemize}
\item
  \textbf{Homoscedasticity (equal variance):}\\
  Levene's test across Genre × YearGroup gave an F-statistic of 34.01
  (\emph{p} = \textless{} 0.001).\\
  Since the test was significant, we conclude residuals are
  heteroscedastic.\\
  To address this, all inference is reported with HC3 robust standard
  errors.
\item
  \textbf{Normality of residuals:}\\
  The Kolmogorov--Smirnov test gave D = 0.042 (\emph{p} = \textless{}
  0.001),\\
  and the Shapiro--Wilk test on a subsample of 5,000 residuals gave W =
  0.990 (\emph{p} = \textless{} 0.001).\\
  Both reject strict normality. However, given the large sample size
  (\textgreater20,000 films), minor deviations are not problematic
  because OLS inference is robust under the central limit theorem.
\item
  \textbf{Multicollinearity:}\\
  The maximum variance inflation factor (VIF) across predictors was
  3.29,\\
  well below the common cutoff of 5. Thus, multicollinearity is not a
  concern.
\end{itemize}

\textbf{Conclusion:}\\
While homoscedasticity and normality assumptions are not fully met,
robust HC3 errors ensure valid inference. Multicollinearity is not an
issue, and the model can be considered well-specified.

To illustrate these diagnostic checks visually, we next examine the
coefficient plot (showing estimated effects with confidence intervals)
and the residuals vs fitted plot (assessing variance and linearity).

\subsubsection{Model Diagnostics}\label{model-diagnostics}

\paragraph{Residuals vs Fitted plot}\label{residuals-vs-fitted-plot}

The residuals vs fitted plot helps assess whether model assumptions of
linearity and equal variance hold:

\begin{itemize}
\tightlist
\item
  Levene's test indicated heteroscedasticity (F = 34.01, \emph{p} =
  \textless{} 0.001), and the plot shows some variation in residual
  spread across fitted values. This was addressed by using HC3 robust
  standard errors.
\item
  Residuals remain centered around zero without strong non-linear
  patterns, supporting the model's linear specification.
\end{itemize}

\subsubsection{Main results table}\label{main-results-table}

Ordinary least squares with HC3 robust standard errors.

\paragraph{Interpretation of Regression
Results}\label{interpretation-of-regression-results}

\begin{itemize}
\item
  \texttt{"Runtime10"} (β₁ = 0.080, \emph{p} = \textless{} 0.001):\\
  For Comedy films in 2011--2015, each additional 10 minutes is
  associated with a 0.080 change in IMDb rating.
\item
  \texttt{"GenreAdventure"} (β₂ = 0.169, \emph{p} = \textless{}
  0.001):\\
  At the average runtime and votes, Adventure films are rated 0.169
  points higher than Comedy.
\item
  \texttt{"GenreAction"} (β₃ = -0.362, \emph{p} = \textless{} 0.001):\\
  At the average runtime and votes, Action films are rated 0.362 points
  lower than Comedy.
\item
  \texttt{"year\_group2016-2020"} (β₄ = -0.068, \emph{p} = \textless{}
  0.001):\\
  At the average runtime and votes, films released in 2016--2020 are
  rated 0.068 points lower than those in 2011--2015.
\item
  \texttt{"Runtime10:GenreAdventure"} (β₅ = -0.045, \emph{p} = 0.006):\\
  The runtime--rating slope for Adventure vs.~Comedy is 0.045 points
  flatter per +10 minutes.
\item
  \texttt{"Runtime10:GenreAction"} (β₆ = 0.015, \emph{p} = 0.159):\\
  The runtime--rating slope for Action vs.~Comedy is 0.015 points
  steeper per +10 minutes.
\item
  \texttt{"Runtime10:year\_group2016-2020"} (β₇ = 0.028, \emph{p} =
  0.004):\\
  The runtime--rating slope in 2016--2020 vs.~2011--2015 is 0.028 points
  steeper per +10 minutes.
\item
  \texttt{"logVotes"} (β₈ = 0.232, \emph{p} = \textless{} 0.001):\\
  A tenfold increase in votes (log₁₀ + 1) is associated with a 0.232
  change in rating.
\end{itemize}

\textbf{Slopes by group (per +10 minutes):}

\begin{itemize}
\tightlist
\item
  Comedy, 2011--2015: 0.080\\
\item
  Adventure, 2011--2015: 0.035\\
\item
  Action, 2011--2015: 0.096\\
\item
  Comedy, 2016--2020: 0.109
\end{itemize}

This table reports the estimated coefficients with HC3 robust standard
errors; the accompanying coefficient plot below provides a visual
summary of these effects and their 95\% confidence intervals.

\paragraph{Coefficient plot}\label{coefficient-plot}

The coefficient plot summarizes the estimated regression effects with
95\% confidence intervals. This makes it clear which predictors are
significantly different from zero and in what direction:

\subsection{Conclusions and
Recommendations}\label{conclusions-and-recommendations}

\subsubsection{Conclusions}\label{conclusions}

This study examined how movie runtime influences IMDb ratings and
whether this relationship is moderated by genre (Comedy, Adventure,
Action) and release period (2011--2015 vs.~2016--2020). Using a dataset
of over 20,000 films, we estimated an OLS regression with interaction
terms, controlling for audience size through log-transformed vote
counts.

This led to the following key findings:

\begin{itemize}
\tightlist
\item
  \textbf{Runtime effect}: For Comedy films released in 2011--2015 (the
  reference group), runtime had only a modest effect on IMDb ratings.
  The slope varied across genres and periods, showing that audience
  tolerance for longer runtimes depends on context.\\
\item
  \textbf{Genre differences}: Action films were rated significantly
  lower than Comedy, while Adventure films were rated slightly higher.
  Importantly, both Action and Adventure showed different
  runtime--rating slopes compared to Comedy, suggesting genre
  conventions influence how audiences perceive film length.\\
\item
  \textbf{Period differences}: Films released in 2016--2020 were
  generally rated lower than those from 2011--2015, with some evidence
  that the runtime effect also shifted in this later period. This aligns
  with the rise of streaming and potential changes in audience
  expectations.\\
\item
  \textbf{Votes as control}: The number of audience votes was a strong
  predictor, confirming that widely viewed films tend to receive higher
  and more stable ratings.
\end{itemize}

Overall, runtime does matter, but its influence is conditional on genre
and release period. Adventure audiences appear more tolerant of longer
runtimes than Comedy audiences, while Action audiences penalize longer
runtimes more strongly. After 2015, movies overall received slightly
lower ratings, indicating shifting audience preferences in the streaming
era.

\subsubsection{Recommendations}\label{recommendations}

Based on these results, we suggest the following recommendations:

\begin{itemize}
\tightlist
\item
  \textbf{Film studios}:

  \begin{itemize}
  \tightlist
  \item
    For \emph{Action} films, keeping runtimes tighter may help maintain
    audience satisfaction, as longer runtimes appear to reduce
    ratings.\\
  \item
    \emph{Adventure} films can sustain longer runtimes, which may
    support more complex storytelling without penalizing ratings.\\
  \item
    \emph{Comedies} remain best received at shorter, more conventional
    lengths.
  \end{itemize}
\item
  \textbf{Streaming platforms}:

  \begin{itemize}
  \tightlist
  \item
    Since the 2016--2020 period shows lower ratings overall, platforms
    may need to adjust recommendation algorithms or promotional
    strategies to counteract declining audience evaluations.\\
  \item
    Tailoring recommendations by runtime preferences (e.g., shorter
    comedies, longer adventures) could improve user satisfaction.
  \end{itemize}
\item
  \textbf{Cinemas and marketing teams}:

  \begin{itemize}
  \tightlist
  \item
    Marketing should emphasize ``epic scale'' and immersion for longer
    Adventure films, while highlighting pacing and efficiency for Action
    and Comedy.\\
  \item
    Runtime can be a strategic element in positioning films to target
    audiences.
  \end{itemize}
\end{itemize}

\subsubsection{Limitations and Future
Research}\label{limitations-and-future-research}

This study has several limitations:

\begin{itemize}
\tightlist
\item
  The analysis focused only on three genres (Comedy, Action, Adventure),
  excluding others such as Drama or Horror that may behave
  differently.\\
\item
  IMDb ratings, while widely used, are subject to self-selection bias
  and may not fully reflect general audience opinion.\\
\item
  The period split (2011--2015 vs.~2016--2020) captures the rise of
  streaming but does not account for other industry shifts (e.g.,
  franchise dominance, COVID-19 disruptions).
\end{itemize}

Future research could expand to other genres, platforms, and audience
measures (e.g., box office revenues, streaming completion rates) to
build a fuller picture of how runtime affects audience perception in
different contexts.

\end{document}
